% !TEX TS-program = xelatex
% !TEX encoding = UTF-8 Unicode
% !Mode:: "TeX:UTF-8"

\documentclass{resume}
\usepackage{zh_CN-Adobefonts_external} % Simplified Chinese Support using external fonts (./fonts/zh_CN-Adobe/)
% \usepackage{NotoSansSC_external}
% \usepackage{NotoSerifCJKsc_external}
% \usepackage{zh_CN-Adobefonts_internal} % Simplified Chinese Support using system fonts
\usepackage{linespacing_fix} % disable extra space before next section
\usepackage{cite}

\begin{document}
\pagenumbering{gobble} % suppress displaying page number

\name{晏成昊}

\basicInfo{
  \email{chenghao_yan@bupt.edu.cn} \textperiodcentered\ 
  \phone{(+86) 186-3543-6558} \textperiodcentered\}
 
\section{\faGraduationCap\  EDUCATION}
\datedsubsection{\textbf{北京邮电大学}, 北京}{2016 -- 至今}
\textit{在读大四本科生}\ 计算机科学与技术

\section{\faUsers\ EXPERIENCE}
\datedsubsection{\textbf{宠物小精灵对战系统设计} }{2018 -- 2019}
\role{C++, QT}{学校大作业}
\begin{itemize}
  \item 实现了一个基于QT、TCP/IP的对战系统
  \item 使用Mysql实现对用户数据的存储
\end{itemize}

\datedsubsection{\textbf{学生信息管理系统}}{2018 -- 2019}
\role{汇编语言}{学校大作业}
\begin{itemize}
  \item 使用汇编语言实现简单的学生管理系统
\end{itemize}

\datedsubsection{\textbf{情感分类任务}}{2018 -- 2019}
\role{\tensorflow, Python}{学校大作业}
\begin{itemize}
  \item 使用tensorflow框架训练一个情感分类器
  \item 准确率达到75%,使用attention机制后准确率提升至80%
\end{itemize}


% Reference Test
%\datedsubsection{\textbf{Paper Title\cite{zaharia2012resilient}}}{May. 2015}
%An xxx optimized for xxx\cite{verma2015large}
%\begin{itemize}
%  \item main contribution
%\end{itemize}

\section{\faCogs\ SKILL}
% increase linespacing [parsep=0.5ex]
\begin{itemize}[parsep=0.5ex]
  \item 编程语言: C、Python、C++
  \item 平台: Linux
  \item 工具: tensorflow、 pytorch
\end{itemize}

\section{\faHeartO\ 获奖情况}
\datedline{校二等奖学金}{2018}
\datedline{校三好学生}{2018}
\datedline{优秀团员}{2018}
\datedline{校二等奖学金}{2017}
\datedline{校优秀班干部}{2018}

%% Reference
%\newpage
%\bibliographystyle{IEEETran}
%\bibliography{mycite}
\end{document}
